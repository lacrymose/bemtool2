\documentclass[a4paper,11pt]{article}

\usepackage{a4wide,amsmath,amssymb,amsthm}
\usepackage{mathrsfs}

\newcommand{\nc}{\newcommand}
\nc{\dsp}{\displaystyle}
\nc{\mrm}{\mathrm}
\nc{\mL}{\mathrm{L}}
\nc{\mH}{\mathrm{H}}
\nc{\mD}{\mathrm{D}}
\nc{\mB}{\mathrm{B}}
\nc{\R}{\mathbb{R}}
\nc{\C}{\mathbb{C}}
\nc{\lbr}{\lbrack}
\nc{\rbr}{\rbrack}
\nc{\bn}{\boldsymbol{n}}
\nc{\bx}{\boldsymbol{x}}
\nc{\by}{\boldsymbol{y}}
\nc{\Green}{\mathscr{G}}
\nc{\SL}{\mathrm{SL}}
\nc{\DL}{\mathrm{DL}}
\nc{\fre}{\mathfrak{e}}

\date{}
\author{Xavier Claeys et Pierre Marchand}
\title{Analytic expression of potential operators for Helmholtz equation in circular and spherical geometries}

\begin{document}

\maketitle


\section*{Analytic solutions in 2-D}

In this section, we give detailed expression of the boundary integral operators associated to a circle in 2-D.
Here we consider an Helmholtz equation   $-\Delta u -\kappa^{2}u = 0$ with outgoing radiation condition.
The corresponding Green kernel is given by
$$
\left\{\begin{array}{l}
-\Delta \Green - \kappa^{2}\Green = \delta_{0}\quad\textrm{in}\;\R^{2}\\[5pt]
\lim_{\rho\to +\infty}\int_{\partial\mrm{D}}\vert \partial_{\rho}\Green - \imath\kappa \Green\vert^{2}d\sigma = 0\\[5pt]
\textrm{where}\quad  \Green(\bx) = \frac{\imath}{4}H_{0}^{(1)}(\kappa\vert \bx\vert)
\end{array}\right.
$$
Let us denote $\mrm{D}\subset\R^{2}$ the disk with center $0$ and radius $\rho$, and $\Gamma = \partial\mD$.
We denote $\gamma_{D}:\mH^{1}(\mD)\to \mH^{1/2}(\Gamma)$ the interior Dirichlet trace defined by $\gamma_{D}(u):= u\vert_{\Gamma}$
for any $u\in \mathscr{C}^{0}(\overline{\mD})$, and  $\gamma_{N}:\mH^{1}(\Delta,\mD)\to \mH^{-1/2}(\Gamma)$ the interior
Neumann trace defined by $\gamma_{N}(u) := \bn\cdot\nabla u\vert_{\Gamma} = \partial_{r}u\vert_{\Gamma}$ where $\bn$ refers to
the normal vector field ditrected toward the exterior of $\mD$. We define $\gamma_{D,c},\gamma_{N,c}$ in the
same manner, except that the traces are taken from the exterior of $\mD$. Finally, we set
$$
\begin{array}{ll}
\{\gamma_{D}\}:= (\gamma_{D}+\gamma_{D,c})/2 &  \{\gamma_{N}\}:= (\gamma_{N}+\gamma_{N,c})/2\\[10pt]
\lbrack\gamma_{D}\rbr:=\gamma_{D}-\gamma_{D,c} & \lbrack\gamma_{N}\rbr:=\gamma_{N}-\gamma_{N,c}.
\end{array}
$$
Let us introduce the layer potentials associated to the interior of the disc $\mD$.
For any trace $v\in\mH^{1/2}(\Gamma), p\in\mH^{-1/2}(\Gamma)$, their explicit expression
is given by:
\begin{equation}\label{PotentialOperators}
\begin{array}{l}
\dsp{ \SL(p)(\bx):=\int_{\Gamma}\Green(\bx-\by)p(\by) d\sigma(\by), }\\[10pt]
\dsp{ \DL(p)(\bx):=\int_{\Gamma}\bn(\by)\cdot(\nabla\Green)(\bx-\by)p(\by) d\sigma(\by). }
\end{array}
\end{equation}
We are going to provide a completely explicit expression of these operators in terms
of Fourier harrmonics and Bessel functions. Set $\fre_{n}(\theta) = \exp(\imath n\theta)$.
We have
$$
\SL(\fre_{n})(r,\theta) =
\left\{\begin{array}{ll}
\dsp{\imath\rho\frac{\pi}{2}H^{(1)}_{\vert n\vert}(\kappa \vert \bx \vert )J_{\vert n\vert}(\kappa \rho)\fre_{n}(\theta)\phantom{\imath\kappa} } & \textrm{for}\quad \vert \bx\vert< \rho\\[15pt]
\dsp{\imath\rho\frac{\pi}{2}J_{\vert n\vert}(\kappa \vert \bx \vert )H^{(1)}_{\vert n\vert}(\kappa \rho)\fre_{n}(\theta)\phantom{\imath\kappa} } & \textrm{for}\quad \vert \bx\vert>\rho
\end{array}\right.
$$
and
$$
\DL(\fre_{n})(r,\theta) =
\left\{\begin{array}{ll}
\dsp{ -\imath\kappa \rho\frac{\pi}{2}H^{(1)'}_{\vert n\vert}(\kappa \rho)J_{\vert n\vert}(\kappa \vert \bx \vert )\fre_{n}(\theta) } & \textrm{for}\quad \vert \bx\vert< \rho\\[15pt]
\dsp{ -\imath\kappa \rho \frac{\pi}{2}J_{\vert n\vert}'(\kappa \rho )H^{(1)}_{\vert n\vert}(\kappa \vert \bx \vert)\fre_{n}(\theta) } & \textrm{for}\quad \vert \bx\vert>\rho
\end{array}\right.
$$
\quad\\[5pt]
From these formulas, it is  clear that $\int_{\Gamma}\overline{\fre}_{p}\gamma_{N}\SL(\fre_{n}) d\sigma = \int_{\Gamma}\overline{\fre}_{p}\gamma_{D}\SL(\fre_{n}) d\sigma = 0$
as well as $\int_{\Gamma}\overline{\fre}_{p}\gamma_{N}\DL(\fre_{n}) d\sigma = \int_{\Gamma}\overline{\fre}_{p}\gamma_{D}\DL(\fre_{n}) d\sigma = 0$ for $p\neq n$. In the
case $p=n$, we have on the one hand
$$
\begin{array}{l}
\dsp{ \int_{\Gamma}\overline{\fre}_{n}\{\gamma_{D}\}\cdot\SL(\fre_{n}) d\sigma = \imath \rho^2 \pi^{2} H^{(1)}_{\vert n\vert}(\kappa \rho ) J_{\vert n\vert}(\kappa \rho) }\\[10pt]
\dsp{ \int_{\Gamma}\overline{\fre}_{n}\{\gamma_{N}\}\cdot\DL(\fre_{n}) d\sigma = -\imath \kappa^{2} \rho^2 \pi^{2} H^{(1)'}_{\vert n\vert}(\kappa \rho) J_{\vert n\vert}'(\kappa \rho) }\\[10pt]
\end{array}
$$
and on the other hand
$$
\begin{array}{l}
\dsp{ \int_{\Gamma}\overline{\fre}_{n}\{\gamma_{N}\}\cdot\SL(\fre_{n}) d\sigma = +\rho^2 \imath\kappa\frac{\pi^{2}}{2}
\big(\; H^{(1)}_{\vert n\vert}(\kappa \rho )J_{\vert n\vert}'(\kappa \rho )+H^{(1)'}_{\vert n\vert}(\kappa \rho )J_{\vert n\vert}(\kappa \rho)\; \big) }\\[10pt]
\dsp{ \int_{\Gamma}\overline{\fre}_{n}\{\gamma_{D}\}\cdot\DL(\fre_{n}) d\sigma = -\imath\kappa \rho^2 \frac{\pi^{2}}{2}
\big(\; H^{(1)}_{\vert n\vert}(\kappa)J_{\vert n\vert}'(\kappa)+H^{(1)'}_{\vert n\vert}(\kappa)J_{\vert n\vert}(\kappa)\; \big) }\\[10pt]
\end{array}
$$

\section*{Analytic solutions in 3-D}

Now we consider the same problem but, this time, for an Helmholtz equation in 3-D with a boundary $\Gamma = \mathbb{S}^{2}$
the unit sphere. The corresponding outgoing radiating kernel satisfies the equations
$$
\left\{\begin{array}{l}
-\Delta \Green - \kappa^{2}\Green = \delta_{0}\quad \textrm{in}\;\R^{3}\\[5pt]
\lim_{\rho\to +\infty}\int_{\partial\mB_{\rho}}\vert \partial_{\rho}\Green - \imath \kappa \Green\vert^{2} d\sigma_{\rho} = 0\\[5pt]
\textrm{where}\quad \Green(\bx) = \exp(\imath\vert \bx\vert)/(4\pi\vert \bx\vert)
\end{array}\right.
$$
Let $\mB\subset \R^{3}$ refer to the ball of center $\mathbf{0}$ and radius $1$, and $\Gamma = \partial\mB$. We use
the same notations regarding trace operators compared to the previous section. in particular, $\bn$ shall refer to the
unit vector normal to $\Gamma$ :directed toward the exterior of $\mB$. Once again, we define the potential operators
by (\ref{PotentialOperators})

\quad\\
Let us recall that any function $u\in \mL^{2}(\Gamma)$ can be decomposed on the orthonormal basis
of spherical harmonics  $\mrm{Y}_{l}^{m}, l\geq 0,\vert m\vert\leq l$ as follows
$$
\begin{array}{l}
\dsp{ u(\theta,\varphi) = \sum_{l=0}^{+\infty}\sum_{-l\leq m\leq +l} u_{l,m}\mrm{Y}_{l}^{m}(\theta,\varphi)
\quad \textrm{where}\quad u_{l,m}:=\int_{\Gamma}u(\sigma) \overline{\mrm{Y}}_{l}^{m}(\sigma) d\sigma   }\\[5pt]

\dsp{ \mrm{Y}_{l}^{m}(\theta,\varphi):= (-1)^{m}\sqrt{\frac{l+1/2}{2\pi}\, \frac{(l-m)!}{(l+m)!}   }\;\mrm{P}_{l}^{m}(\cos\theta) \exp(\imath m\varphi)  }.
\end{array}
$$
Here the $\mrm{P}_{l}^{m}(z)$ are associated Legendre functions. Define the spherical Bessel and Hankel functions $j_{l},h_{l}^{(1)}$
like in paragraph 10.47 of \cite{MR2723248}. The single layer and double layer potentials admit explicit expressions
in terms of the spherical harmonics. On the one hand\\
$$
\SL(\mrm{Y}_{l}^{m})(\rho,\sigma) =
\left\{\begin{array}{ll}
\dsp{\imath\kappa\, j_{l}(\kappa\rho)h^{(1)}_{l}(\kappa)\mrm{Y}_{l}^{m}(\theta,\varphi)   } & \textrm{for}\;\;\rho<1\\[10pt]
\dsp{\imath\kappa\, h^{(1)}_{l}(\kappa\rho)j_{l}(\kappa)\mrm{Y}_{l}^{m}(\theta,\varphi)   } & \textrm{for}\;\;\rho>1.
\end{array}\right.
$$
And on the other hand we have
$$
\DL(\mrm{Y}_{l}^{m})(\rho,\sigma) =
\left\{\begin{array}{ll}
\dsp{-\imath\kappa^{2}\, j_{l}(\kappa\rho)h^{(1)'}_{l}(\kappa)\mrm{Y}_{l}^{m}(\theta,\varphi)   } & \textrm{for}\;\;\rho<1\\[10pt]
\dsp{-\imath\kappa^{2}\, h^{(1)}_{l}(\kappa\rho)j_{l}'(\kappa)\mrm{Y}_{l}^{m}(\theta,\varphi)   } & \textrm{for}\;\;\rho>1.
\end{array}\right.
$$
Note that, like in the  2-D problem of the previous paragraph, we have $\int_{\Gamma}\overline{\mrm{Y}}_{p}^{q}\gamma_{D}\SL(\mrm{Y}_{l}^{m})d\sigma =$
$\int_{\Gamma}\overline{\mrm{Y}}_{p}^{q}\gamma_{N}\SL(\mrm{Y}_{l}^{m})d\sigma = 0$ for $p\neq l$ or $q\neq m$, and similar identity holds for
$\DL$. Hence we have the following expressions for the entries of the Calderon projector
$$
\begin{array}{l}
\dsp{ \int_{\Gamma}\overline{\mrm{Y}}_{l}^{m}\{\gamma_{D}\}\SL(\mrm{Y}_{l}^{m}) d\sigma = \imath\kappa \,j_{l}(\kappa)h_{l}^{(1)}(\kappa), }\\[10pt]
\dsp{ \int_{\Gamma}\overline{\mrm{Y}}_{l}^{m}\{\gamma_{N}\}\DL(\mrm{Y}_{l}^{m}) d\sigma = -\imath\kappa^{3} \,j_{l}'(\kappa)h_{l}^{(1)'}(\kappa). }
\end{array}
$$
As well as
$$
\begin{array}{l}
\dsp{ \int_{\Gamma}\overline{\mrm{Y}}_{l}^{m}\{\gamma_{N}\}\SL(\mrm{Y}_{l}^{m}) d\sigma = +\imath\frac{\kappa^{2}}{2}\big( \,j_{l}'(\kappa)h_{l}^{(1)}(\kappa) +  j_{l}(\kappa)h_{l}^{(1)'}(\kappa)\,\big), }\\[10pt]
\dsp{ \int_{\Gamma}\overline{\mrm{Y}}_{l}^{m}\{\gamma_{D}\}\DL(\mrm{Y}_{l}^{m}) d\sigma = -\imath\frac{\kappa^{2}}{2}\big( \,j_{l}'(\kappa)h_{l}^{(1)}(\kappa) +  j_{l}(\kappa)h_{l}^{(1)'}(\kappa)\,\big). }
\end{array}
$$


\bibliography{biblio}
\bibliographystyle{plain}
\end{document}
